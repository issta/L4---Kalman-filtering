%!TEX root = /Users/kevin/SkyDrive/KTH Work/Period 4 2014/GNSS/Labs/L4 - Kalman filtering/main.tex

\section{Methodology} 

% (fold)
\label{sec:methodology}

% Describe the methods and instruments used to solve the problem. 
Matlab was used to solve for the following parameters and all of the following equations were taken from the technical report: Kalman Filtering.\cite{milanGPS} 
The steps and equations from the Kalman filtering algorithm are documented in the Matlab code. 
We start off with the matrix differential equation,
\begin{equation}
	\dot{\+x} (t) = \+F (t) x (t) + \+G (t) \+u (t) 
	\label{eq:eq1}
\end{equation} % (eq:eq1)
where $\+x = [${\rmfamily e~n~v$_e$~v$_n$}$]^{\mathrm{T}}$, the east and north positions and their velocity components. \+F, \+G, and \+u are given by,
\begin{equation}
	\begin{array}{ccc}
		\+F= \left[
		\begin{matrix}
			0 & 0 & 1 & 0 \\
			0 & 0 & 0 & 1 \\
			0 & 0 & 0 & 0 \\
			0 & 0 & 0 & 0 \\
		\end{matrix}
		\right] & 
		\+G = \left[
		\begin{matrix}
			0 & 0 \\
			0 & 0 \\
			1 & 0 \\
			0 & 1 \\
		\end{matrix}
		\right] & 
		\+u = \left[
		\begin{matrix}
			\omega_{ae} \\
			\omega_{an} \\
		\end{matrix}
		\right] 
	\end{array}
\end{equation}
We do not know \+u because it represents white noise.  The discrete solution to Equation~\eqref{eq:eq1} is
\begin{equation}
	\+x_k = \+T_{k-1,k}\+x_{k-1} + \+w_{k-1,k}
	\label{eq:discreteSolutionToeq1}
\end{equation} % (eq:discretesolutiontoeq1)
where values of $k$ represent discretization of time. The transition matrix, $\+T_{k-1,k}$, can be approximated by,
\begin{equation}
	\+T_{k-1,k} = \+I + \+F_{k} \Delta t = \left[
			\begin{matrix}
				1 & 0 & \Delta t & 0 \\
				0 & 1 & 0           & \Delta t \\
				0 & 0 & 1           & 0 \\
				0 & 0 & 0           & 1 \\
			\end{matrix}
			\right]
	\label{eq:transitionMatrix}
\end{equation} % (eq:transitionmatrix)
The covariance of $\+w_k$, $\+Q_k$, is evaluated by use of the power spectral density (PSD) of the random acceleration, which are initially given. We can now represent the covariance of $\+w_k$ as,
\begin{equation}
	\+Q_k = \+Q_{G} \Delta t + (\+F \+Q_{G} + \+Q_{G} \+F^{\mathrm{T}})\frac{\Delta t^2}{2}
	+ \+F \+Q_{G} \+F^{\mathrm{T}} \frac{\Delta t^{3}}{3}
	\label{eq:eq12}
\end{equation} % (eq:covarianceqk)
where
\begin{equation}
	\begin{array}{cc}
		\+Q_{G} = \+G\+Q\+G^{\mathrm{T}} & \+Q = \left[
		\begin{matrix}
		q_{ae} 	& 0 		\\
		0		& q_{an} 	\\
		\end{matrix}
		\right]
	\end{array}
	\label{eq:eq11}
\end{equation} % (eq:q_g)


\subsection{Kalman Filter} % (fold)
\label{sub:kalman_filter}
The main steps of the discrete Kalman Filter algorithm are:
\begin{enumerate}
	\item Initialization:
	\begin{equation}
		\+x_0,~~\+Q_{x0} = \mathrm{var}[x_0]
		\label{eq:eq15}
	\end{equation} % (eq:eq15)
	\item Time propagation
	\begin{equation}
		\+x_k^- = \+T_{k-1,k}\+x_{k-1},~~\+Q_{x,k}^-= \+T_{k-1,k}\+Q_{x,k-1}\+T_{k-1,k}^{\mathrm{T}} + \+Q_k
		\label{eq:eq16}
	\end{equation} % (eq:eq16)
	\item Gain calculation:
	\begin{equation}
		\+K_k = \+Q_{x,k}^-\+H_k^{\mathrm{T}}[\+R_k+\+H_k \+Q_{x,k}^-\+H_k^{\mathrm{T}}]^{-1}
		\label{eq:eq17}
	\end{equation} % (eq:eq17)
	\item Measurement update
	\begin{equation}
		\+x_k = \+x_k^-+\+K_k[\tilde{L}_k-\+h_k(x_k^-)]
		\label{eq:eq18}
	\end{equation} % (eq:eq18)
	\item Covariance update
	\begin{equation}
		\+Q_{x,k} = [\+I-\+K_k \+H_k]\+Q_{x,k}^-
		\label{eq:19}
	\end{equation} % (eq:19)
\end{enumerate}
% subsection kalman_filter (end)
\subsection{Smoothing} % (fold)
\label{sub:smoothing}
After the Kalman Filter algorithm, we then smooth the results using the smoothed estimations for previous epochs,
\begin{equation}
	\hat{\+x}_k = \+x_k + \+D_k [\hat{\+x}_{k+1} - \+x_{k+1}^-]
	\label{eq:eq28}
\end{equation} % (eq:eq28)
\begin{equation}
	\+D_k = \+Q_{x,k}\+T_{k-1,k}^{T}(Q_{x,k+1}^-)^{-1}
	\label{eq:eq29}
\end{equation} % (eq:eq29)
with a covariance matrix of,
\begin{equation}
	\hat{\+Q}_{x,k} = \+Q_{x,k}+\+D_{k}[\hat{\+Q}_{x,k+1}-\+Q_{x,k+1}^-]\+D_k^{\mathrm{T}}
	\label{eq:eq30}
\end{equation} % (eq:eq30)
% subsection smoothing (end)
